%% Copyright 2013 BarD Software s.r.o
%
% This work may be distributed and/or modified under the
% conditions of the LaTeX Project Public License version 1.3c,
% available at http://www.latex-project.org/lppl/.

% === Стандартный moderncv и стандартные темы ===
% === Компилируются pdflatex'ом               ===
% \documentclass[11pt,a4paper]{moderncv}
% \moderncvtheme[grey]{classic}
% \usepackage[utf8]{inputenc}
% \usepackage[T2A]{fontenc}
% \usepackage[english,russian]{babel}

% === moderncv модифицированный для использования с XeLaTeX и фирменная тема ===
\documentclass[11pt,a4paper]{moderncv-xetex}
\moderncvtheme[grey]{papeeria}

\usepackage{xcolor}
\usepackage[a4paper,margin=1in]{geometry}

% Шапка
\firstname{Kristina}\familyname{Fedorenko}
\address{}{Saint Petersburg}
\mobile{+79633494107}
\email{kr.fedorenko@gmail.com}
\quote{A specialist in data mining and statistics}

\begin{document}
\maketitle

% ОПЫТ РАБОТЫ --------------------------------------------------------
\section{Employment}
\cventry{2012--to present}{Saint Petersburg}{AdRiver}{Senior analyst}
    {
    \begin{itemize}
    \item{Preprocessing of data and inventing input variables for data mining models }
    \item{Designing and implementing data mining models for online advertising}
    \item{Modification and adjustment of these algorithms/models according to different types of input data and the particular objectives that the models were aimed at reaching}
    \end{itemize}
    }
% ТЕХНИЧЕСКИЕ НАВЫКИ -------------------------------------------------
\section{Технические навыки}
\cvline{Языки\\ программирования}{Использую в повседневной работе: Brainfuck\newline
Имею представление: Python, Java, C\#}
\cvline{Операционные системы}{BolgenOS, Windows 99}

% ОБРАЗОВАНИЕ --------------------------------------------------------
\section{Образование}
\cventry{2008\,--\,2013}{Специалист}{НИИЧаВо}{Соловец}{}{кафедра экспериментальной евгеники}  % arguments 3 to 6 are optional

\newpage
% ДИПЛОМНАЯ РАБОТА ---------------------------------------------------
\section{Дипломная работа}
\cvline{Название}{\emph{Электронно-механическое  устройство для решения научных,
социологических и иных проблем}}
\cvline{Науч. руководитель}{Амвросий Амбруазович Выбегалло}
\cvline{Описание}{\small Высочайшее достижение нейтронной мегалоплазмы!
Ротор поля наподобие дивергенции градуирует  себя  вдоль  спина и там, внутре, обращает материю
вопроса в спиритуальные электрические вихри, из  коих  и возникает синегдоха отвечания\dots}

% ИНОСТРАННЫЕ ЯЗЫКИ --------------------------------------------------
\section{Иностранные языки}
\cvline{Английский}{Свободно читаю технические тексты и могу поддерживать разговор}
\cvline{Испанский}{Свободно отличаю от португальского и от каталанского}

% ХОББИ --------------------------------------------------------------
\section{Хобби}
\cvline{Геокешинг}{Люблю лазать по болотам в поисках кладов}
\cvline{Прокрастинация}{Люблю лазать по сайтам в поисках мотиваторов и демотиваторов}
\cvline{Скалолазание}{Просто люблю лазать}

% РЕКОМЕНДАЦИИ -------------------------------------------------------
\section{Рекомендации}
\cvline{А.И. Привалов}{Заведующий вычислительным центром НИИЧаВо.\newline \href{mailto:privalov@nii4avo.ru}{privalov@nii4avo.ru}}
\cvline{А. Балаганов}{Уполномоченный по копытам ООО <<Рога и Копыта>>\newline \href{mailto:balaganov@horns-and-hooves.com}{balaganov@horns-and-hooves.com}}
\cvline{С.М. Брин}{Computer scientist, Internet entrepreneur\newline \href{mailto:theotherboss@google.com}{theotherboss@google.com}}
\end{document}
